\chapter{Introduction}
\label{cha:Introduction}

Activity recognition is a big topic in building context aware systems. As wearable devices such as smartphones and smartwatches are usually with their user all day, they are predestined for tracking the user's activity. The integrated variety of sensors provide data which can be interpreted using machine learning methods. 

In this thesis known algorithms are implemented in the beginning in order to detect high-level activities like standing, walking, running and stair climbing. As most of them are already part of the operating systems' provided APIs, the purpose for this is to gain experience in proven and tested approaches for feature extraction. With this knowledge it will finally be tried to detect the entering of a room by analyzing sensors like a barometer. 

As a use case the knowledge of high-level activities could be used for context aware systems in which the location within a building could be estimated. 

In the course of this thesis it will be investigated in the following questions:
\begin{itemize}
	\item How reliable can the entering of a room be detected? Under which circumstances?
	\item How well do known approaches for activity recognition lead to a satisfying result?
\end{itemize}

